%%=============================================================================
%% Samenvatting
%%=============================================================================

% TODO: De "abstract" of samenvatting is een kernachtige (~ 1 blz. voor een
% thesis) synthese van het document.
%
% Deze aspecten moeten zeker aan bod komen:
% - Context: waarom is dit werk belangrijk?
% - Nood: waarom moest dit onderzocht worden?
% - Taak: wat heb je precies gedaan?
% - Object: wat staat in dit document geschreven?
% - Resultaat: wat was het resultaat?
% - Conclusie: wat is/zijn de belangrijkste conclusie(s)?
% - Perspectief: blijven er nog vragen open die in de toekomst nog kunnen
%    onderzocht worden? Wat is een mogelijk vervolg voor jouw onderzoek?
%
% LET OP! Een samenvatting is GEEN voorwoord!

%%---------- Nederlandse samenvatting -----------------------------------------
%
% TODO: Als je je bachelorproef in het Engels schrijft, moet je eerst een
% Nederlandse samenvatting invoegen. Haal daarvoor onderstaande code uit
% commentaar.
% Wie zijn bachelorproef in het Nederlands schrijft, kan dit negeren, de inhoud
% wordt niet in het document ingevoegd.

\IfLanguageName{english}{%
\selectlanguage{dutch}
\chapter*{Samenvatting}
\selectlanguage{english}
}{}

%%---------- Samenvatting -----------------------------------------------------
% De samenvatting in de hoofdtaal van het document

\chapter*{\IfLanguageName{dutch}{Samenvatting}{Abstract}}
Het is belangrijk dat musea weten op welke manieren zij virtual reality kunnen implementeren om de ervaring van het museumbezoek naar een hoger niveau te tillen.
Daarom dat studie heeft onderzocht welke technologieen er beschikbaar zijn en welke technologie de beste is voor de use case 'Het vertellen van een levensverhaal van een persoon in musea'.

Eerst is er longlist opgesteld waarin al deze technologieën beschreven zijn met hun voor- en nadelen.
Uit deze longlist is gebleken dat augmented reality de meeste toepasselijke VR-technologie is omdat deze een goede balans heeft tussen realisme, interactiviteit en bruikbaarheid.
In tegenstelling tot virtual en mixed reality is augmented reality makkelijker te implementeren en ook veel goedkoper voor het museum.

Omdat er binnenin augmented reality veel verschillende frameworks beschikbaar zijn is er ook een shortlist met een vergelijking van de uniekste frameworks.
Als resultaat van dit onderzoek blijkt dat Vuforia en ARCore allebei zeer goed scoren op de belangrijke aspecten van augmented reality zoals plane detection en image recognition maar dat ARCore toch net iets beter scoort op het vlak van performance alsook bij het herkennen van afbeeldingen in donkere ruimtes.

Deze studie bevat ook een applicatie, ontwikkelt met ARCore, die alle requirements van de use case bevat. De applicatie is zo generiek mogelijk ontworpen wat ervoor zorgt dat vele musea de zelfde applicatie kunnen hergebruiken voor verschillende levensverhalen te vertellen. Als verder onderzoek zou het mogelijk zijn om te kijken op welke manieren deze use case uitbreidbaar is. Het gebruik van beacons om de bezoeker rond te leiden in het museum kan bijvoorbeeld meehelpen aan de kwaliteit van de ervaring.