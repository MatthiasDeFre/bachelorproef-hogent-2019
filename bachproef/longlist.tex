\chapter{Longlist verschillende AR / VR technologieën}
\label{ch:longlist}

Hierin worden de verschillende technologieën overlopen die kunnen gebruikt worden om deze use case te volbrengen.
Bij elke technologie zal er dan worden gezegd waarom deze voldoet of niet voldoet aan de eisen (TODO) van de use case.

\section{Head Mounted Virtual Reality}
Bij head mounted virtual reality wordt er gebruik gemaakt van een HDM (head-mounted-display) die de gebruiker op zijn hoofd zet. Deze headset zal dan een virtuele wereld tonen die volledig aangestuurd is door de applicatie zelf. Deze wereld zal dus geen rekening houden met de echte wereld, dit kan namelijk voor problemen zorgen indien er geen sensor of camera aanwezig is die je tegenhoud om tegen muren te lopen.
\section{Augmented Reality}
\section{360 Degrees Videos}
\section{Mixed Reality}
Ook hierbij wordt er gebruik gemaakt van een HDM. Het grote verschil met virtual reality is dat er bij mixed reality ook rekening wordt gehouden met de echte wereld. Een mixed reality headset heeft namelijk de mogelijkheid om de echte wereld te mappen naar een virtuele wereld a.h.v. de camera die langs de voorkant gemonteerd is.

Een mixed reality headset is ook uitgerust met verschillende sensoren om de positie binnen in de wereld te volgen. Door de combinatie van de mapping en sensoren is het dus niet nodig om extra sensoren te plaatsen om room-scale VR te te gebruiken.
\subsection{Spatial Mapping}
Het algoritme dat gebruikt wordt voor de mapping is spatial mapping. Dit algoritme % TODO CONTENT Uitleg spatial mapping