%%=============================================================================
%% Methodologie
%%=============================================================================

\chapter{Methodologie van het Experiment}
\label{ch:methodologie}

Om de kwaliteit van een augmented reality framework te beoordelen is het experiment in twee delen opgedeeld. Het eerste deel van het experiment zal gaan over hoe goed de frameworks de core van de applicatie kunnen implementeren. Elk framework zal dezelfde lijst met tien images. De bedoeling is dat elk framework zoveel mogelijk images probeert te herkennen. De gebruikte lijst met images kan u vinden in de bijlage. De lijst is opgebouwd uit vijf makkelijke afbeeldingen en vijf moeilijkere afbeeldingen. Om te weten wat een moeilijke of makkelijke afbeelding is gebruikt deze studie volgende criteria.

\begin{itemize}
    \item Veel features
    \item Geen repetitieve features
    \item Goed contrast
\end{itemize} 


Het tweede deel van het experiment zal gaan over de performance van elk framework. Dit wordt gedaan door FPS, RAM gebruik en batterij verbruik met elkaar te vergelijken. Omdat ieder framework gebruik maakt van andere API's moet voor elk van deze een soortgelijke applicatie worden voorzien. Deze applicatie zal een afbeelding herkennen aan hierop een 3d object tonen. Bij het klikken op dit object zal dit veranderen in een ander object. Het is de bedoeling dat bij het verliezen van de tracking van de afbeelding, en hierna terug tracken, het nieuwe object er nog steeds is en niet terug verandert is naar het oude.
%% TODO: Hoe ben je te werk gegaan? Verdeel je onderzoek in grote fasen, en
%% licht in elke fase toe welke stappen je gevolgd hebt. Verantwoord waarom je
%% op deze manier te werk gegaan bent. Je moet kunnen aantonen dat je de best
%% mogelijke manier toegepast hebt om een antwoord te vinden op de
%% onderzoeksvraag.
