%%=============================================================================
%% Inleiding
%%=============================================================================

\chapter{Inleiding}
\label{ch:inleiding}

\section{Probleemstelling}
\label{sec:probleemstelling}

Vele musea organiseren tentoonstellingen rond bekende (historische) personen. Deze tentoonstellingen zijn echter vaak verouderd en evolueren ook niet echt mee met technologische ontwikkelingen. Musea die wel de nieuwste technologische snufjes gebruiken moeten vaak terugvallen op standaardapplicaties die vaak niet specifiek genoeg zijn. Deze studie wil één van deze specifieke use cases (het vertellen van een levensverhaal) oplossen door een lijst op te stellen van de mogelijkheden, met hun voor- en nadelen, die musea hebben om dit te implementeren. Alhoewel de hoofddoelgroep musea met deze use case zijn, probeert de studie een zo'n ruim mogelijk beeld te schetsen van de verschillende mogelijkheden om virtual reality te implementeren in een museum zodat deze lijst ook bruikbaar is voor musea met soortgelijke use cases.

\section{Onderzoeksvraag}
\label{sec:onderzoeksvraag}

\begin{enumerate}
    \item Op welke manieren kan virtual reality worden geïmplementeerd in musea?' geeft aan welke technologieën een museum allemaal kan gebruiken om \acrshort{vr} te implementeren.
    \item 'Hoe kan het levensverhaal van een persoon worden verteld in musea en welke technologie is het meest gepast' gebruikt de meeste passende technologie uit de eerste onderzoeksvraag om zo de mogelijkheden van deze technologie te verwerken in een applicatie.
    \item 'Wat is het beste framework van de gekozen technologie op basis van verschillende factoren, beschreven in \ref{ch:methodologie}' 
\end{enumerate}

\section{Onderzoeksdoelstelling}
\label{sec:onderzoeksdoelstelling}

Als resultaat uit dit onderzoek vloeit een handleiding voort aan de hand van een longlist die musea kunnen gebruiken om op verschillende manieren \acrshort{vr} te implementeren. Alsook is er een voorbeeld van een applicatie die een antwoord geeft op de tweede onderzoeksvraag.

\section{Opzet van deze bachelorproef}
\label{sec:opzet-bachelorproef}

% Het is gebruikelijk aan het einde van de inleiding een overzicht te
% geven van de opbouw van de rest van de tekst. Deze sectie bevat al een aanzet
% die je kan aanvullen/aanpassen in functie van je eigen tekst.

De rest van deze bachelorproef is als volgt opgebouwd:

In Hoofdstuk~\ref{ch:stand-van-zaken} wordt een overzicht gegeven van de stand van zaken binnen het onderzoeksdomein, op basis van een literatuurstudie.

In Hoofdstuk~\ref{ch:longlist} wordt een lijst gegeven van verschillende technologieën die gebruikt kunnen worden om \acrshort{vr} te implementeren en beantwoordt dus de eerste onderzoeksvraag.

In Hoofdstuk~\ref{ch:shortlist} wordt een lijst gegeven van verschillende frameworks over de gekozen technologie uit de longlist en bevat voor elk van deze frameworks een score om zo een antwoord te geven op de tweede onderzoeksvraag.

In Hoofdstuk~\ref{ch:methodologie} wordt de methodologie toegelicht en worden de gebruikte onderzoekstechnieken besproken om een antwoord te kunnen formuleren op het experiment van de derde onderzoeksvraag.

In Hoofdstuk~\ref{ch:experiment} worden de resultaten van het experiment geanalyseerd en besproken.

In Hoofdstuk~\ref{ch:experiment} wordt het prototype van de beste technologie besproken, alsook de ontwikkeling van dit prototype wordt hier besproken.

In Hoofdstuk~\ref{ch:conclusie} tenslotte wordt de conclusie gegeven en een antwoord geformuleerd op de onderzoeksvragen. Daarbij wordt ook een aanzet gegeven voor toekomstig onderzoek binnen dit domein.

