%%=============================================================================
%% Conclusie
%%=============================================================================

\chapter{Conclusie}
\label{ch:conclusie}

%% TODO: Trek een duidelijke conclusie, in de vorm van een antwoord op de
%% onderzoeksvra(a)g(en). Wat was jouw bijdrage aan het onderzoeksdomein en
%% hoe biedt dit meerwaarde aan het vakgebied/doelgroep? Reflecteer kritisch
%% over het resultaat. Had je deze uitkomst verwacht? Zijn er zaken die nog
%% niet duidelijk zijn? Heeft het onderzoek geleid tot nieuwe vragen die
%% uitnodigen tot verder onderzoek?

Net zoals bij andere informaticaontwikkelingen zien we dat virtual reality ook vaak te kampen heeft met te moeten wachten op een andere hard- of software. Zelfs zoiets simpel als augmented reality moet vaak rekening houden met een groot verschil van hardware binnenin smartphones. Virtual en Mixed Reality hebben dan weer veel te dure prijzen omdat het ontwikkelingsproces van deze nog niet geoptimaliseerd is. Het daarom moeilijk te zeggen wat de toekomst juist gaat inhouden voor Virtual Reality wel zien we dat grote bedrijven hiermee experimenteren en al goede resultaten boeken. Echter zal het een hele tijd duren vooraleer deze leuke snufjes betaalbaar zijn voor consumenten of zelfs musea. Dit zorgt wel niet voor dat het onmogelijk is om immersive ervaringen te maken, het kost alleen net iets meer moeite van developers en designers om een applicatie te ontwikkelen die het gepaste niveau van teleprescence opkweekt met de beperkte tooling.

Musea zitten momenteel ook met het probleem dat deze niet echt beschikken over een infrastructuur om de vr-applicaties in te draaien. Het is daarom eigenlijk wel logisch dat augmented reality als beste technologie naar voren komt omdat er hiervoor weinig infrastructuur nodig is. De opkomst van 5G zou vele van deze problemen wel moeten oplossen. De snelheid van 5G is veel sneller dan de huidige verbindingen en zou toestaan om vele software volledig in de cloud te laten draaien waardoor headsets en smartphones de taak van het mappen van een ruimte kunnen overlaten aan de cloud. Dit lost ook meteen latency luik van virtual reality sickness op zoals beschreven in sectie \ref{sec:medical}\autocite{Bastug2017}.

Voor het implementeren van de applicatie maakte het framework niet zoveel uit. Meeste AR frameworks beschikken over dezelfde features waardoor de keuze vrij is. Bij de frameworks gespecialiseerd op één bepaald doelgebied (Vuforia met image recognition en 8th Wall met browserondersteuning) is het wel duidelijk dat deze lager scoren in de andere doelgebieden.