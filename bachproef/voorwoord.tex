%%=============================================================================
%% Voorwoord
%%=============================================================================

\chapter*{Woord vooraf}
\label{ch:voorwoord}

%% TODO:
%% Het voorwoord is het enige deel van de bachelorproef waar je vanuit je
%% eigen standpunt (``ik-vorm'') mag schrijven. Je kan hier bv. motiveren
%% waarom jij het onderwerp wil bespreken.
%% Vergeet ook niet te bedanken wie je geholpen/gesteund/... heeft
Ik heb gekozen voor dit onderwerp omdat VR-applicaties in de toekomst meer een meer zullen voorkomen en dat het belangrijk is dat gebruikers hier al ervaring mee hebben. Grote bedrijven maken momenteel grote vooruitgang op het vlak van deze technologie maar toch vind ik dat de gewone mens ook het recht heeft om gebruik te maken van VR bij alledaagse zaken zoals een museumbezoek. Het is daarom dat ik opzoek ben gegaan naar een manier om zoveel mogelijk mensen de kans te geven om 'Het levensverhaal van een persoon' in VR te ervaren.

Graag wil ik hierbij mijn promotor Buysse Jens, copromotor De Lamper Joris en ook de andere werknemers van het VR-team van In The Pocket bedanken voor hun feedback en om mij bij te staan bij het succesvol opstellen van deze bachelorproef.

Alsook wil ik Vannieuwenhuyse Kaat van het \acrfull{muhka} bedanken voor haar feedback en inbreng waardoor ik de requirements en valkuilen van deze use case heb kunnen opstellen.
