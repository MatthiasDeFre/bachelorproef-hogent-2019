\chapter{Specificaties en Valkuilen}
\label{ch:requirements}

Om alle technologieën gelijk te beoordelen bevat dit hoofdstuk de generieke eisen waaraan de applicatie moet voldoen. Ook de belangrijkste valkuilen zijn een onderdeel van dit hoofdstuk. Deze requirements en valkuilen zijn opgesteld door voorafgaande use cases te analyseren en ook musea te contacteren. 


\subsection{Requirements \acrshort{muhka}}
De grootste hulp hierbij was Vannieuwenhuyse Kaat van het \acrlong{muhka} zij heeft tijdens een gesprek twee belangrijke requirements opgesteld: 

\begin{enumerate}
    \item Toegevoegde waarde hebben voor de gebruiker, het moet meerwaarde bieden en mag niet opdringerig of een vervanging zijn 
    \item Het moet herbruikbaar zijn, de tijd die het museum erin steekt mag niet verloren gaan eens de tentoonstelling voorbij is 
\end{enumerate}

Om waarde toe te voegen is het belangrijk dat de applicatie dient als hulpmiddel voor de bezoeker en niet als enigste manier om de tentoonstelling te ervaren. Herbruikbaarheid kan op verschillende manieren geïmplementeerd worden. Net zoals bij het \acrshort{muhka} zal de applicatie dus beschikken over een archief van alle tentoonstellingen, zelfs als deze niet meer actief zijn in het museum. 

\subsection{Analyse Use Cases}
Iets dat vaak terugkomt bij andere use cases in musea is het gebruikmaken van de bestaande infrastructuur. De integratie van bestaande kunstwerken moet dus ook mogelijk zijn. Of deze kunstwerken fysiek of virtueel aanwezig moeten zijn hangt af van de meeste passende technologie uit de longlist.

Het is ook heel belangrijk dat de applicatie toegankelijk is voor iedereen. De doelgroep van een museum is heel variabel wat ervoor zorgt dat de applicatie en de technologie bruikbaar moet zijn door jong en oud.

Verdere requirements die eigen zijn aan de gekozen technologie zijn te vinden in sectie \ref{sec:arrequirements}.