\chapter{Shortlist frameworks ---GEKOZEN TECHNOLOGIE----}
\label{ch:shortlist}

In dit voorbeeld gebruik ik augmented reality

Eerst zal er een kleine inleiding worden gegeven over verschillende algoritmen die worden gebruikt binnenin augmented reality zoals plane detection, image detection... zodat de gebruiker goed weet wat deze termen betekenen.

Hierin worden de verschillende frameworks van de gekozen technologie uit de longlist overlopen en met elkaar vergeleken. Dit zal gebeuren door aan elk framework een score te geven gebaseerd op een lijst van must-haves en nice-to-haves. Deze zullen dan elk factor hebben die meer (must-have) of minder (nice-to-have) zal doorwegen in de score.

Vb must-have: moet op zoveel mogelijk devices draaien => aantal procent van de devices kan dan de score zijn voor deze must have
OF
een andere manier simpelere manier om dit voor te stellen zou a.h.v. sterren kunnen zijn. Bv 1 ster: draait op ioS, draait op android, draait op andere VR headsets...

Onderste secties zijn voorbeelden indien augmented reality wordt gekozen.

\section{8th Wall}
\section{ARCore}
\section{ARKit}
\section{Vuforia}