%---------- Inleiding ---------------------------------------------------------

\section{Introductie} % The \section*{} command stops section numbering
\label{sec:introductie}

Virtual reality wordt momenteel al op veel verschillende manieren gebruikt in verscheidende sectoren. Dit onderzoek wil nagaan of het ook mogelijk is om virtual reality of augmented reality op een goede manier te gebruiken om het levensverhaal van een persoon te vertellen. Indien dit effectief mogelijk is zal er een long- en shortlist worden gemaakt om de verschillende frameworks en technologieën met elkaar te vergelijken. Aan de hand van de shortlist kan er dan worden gekeken welke technologie de beste is. Met beste wordt hier niet de meest performante bedoeld maar eerder welk prototype het meest toepasbaar is in de gekozen sector.

%---------- Stand van zaken ---------------------------------------------------

\section{State-of-the-art}
\label{sec:state-of-the-art}

Momenteel zijn er al verschillende onderzoeken geweest naar het toepassen van virtual reality in verscheidende sectoren. Ook is er al onderzoek gedaan naar het gebruik van virtual reality in de sectoren die in dit onderzoek aan bod komen, namelijk psychologie \autocite{Wilson2014} en musea \autocite{Jung2016}. Maar geen enkel van deze onderzoeken gaat specifiek over het vertellen van een levensverhaal in virtual reality.
Bij het opzetten van een applicatie in VR is het heel belangrijk dat er rekening wordt gehouden met de sociale aanwezigheid \autocite{Jung2016}. Hiermee wordt er bedoeld dat hoe minder de gebruiker het doorheeft dat hij in een virtuele wereld zit hoe beter de ervaring zal zijn.   

Uit een onderzoek in de psychologie blijkt dat virtual reality positieve effecten kan hebben, maar dat er ook opgepast moet worden voor bepaalde bijeffecten indien de ervaring niet immersief genoeg is \autocite{Wilson2014}.

Bij het toepassen van virtual reality in musea moet er een duidelijk onderscheid worden gemaakt tussen virtual reality en augmented reality. Hoewel virtual reality (met bewegingsvrijheid) een heel immersieve ervaring geeft, zullen veel gebruikers en musea toch augmented reality verkiezen \autocite{Kersten2017}. Dit is omdat het gebruik van AR gepaard zal gaan met een fysiek bezoek aan het museum. Natuurlijk is het ook belangrijk hierbij dat de ervaring leerrijk is. Uit het onderzoek van Davies Andrew G.\textcite{Davies2018} blijkt dat VR inderdaad kan gebruikt worden om iets bij te leren.

Veel mensen worden afgeschrikt door VR ontwikkeling en gebruik door de kostprijs. Alhoewel dit nu een kleiner probleem is door de opkomst van nieuwe VR frameworks zoals WebVR die het ontwikkelen van VR goedkoper maken \autocite{Dibbern2018}. Ook omdat nu bijna alle smartphones uitgerust zijn met de mogelijkheid om VR te tonen is het gebruik van deze applicaties toegankelijker.
% Voor literatuurverwijzingen zijn er twee belangrijke commando's:
% \autocite{KEY} => (Auteur, jaartal) Gebruik dit als de naam van de auteur
%   geen onderdeel is van de zin.
% \textcite{KEY} => Auteur (jaartal)  Gebruik dit als de auteursnaam wel een
%   functie heeft in de zin (bv. ``Uit onderzoek door Doll & Hill (1954) bleek
%   ...'')

%---------- Methodologie ------------------------------------------------------
\section{Methodologie}
\label{sec:methodologie}

Tijdens het onderzoek zal er worden gekeken naar de verschillende manieren om Virtual en Augmented Reality te ontwikkelen waaronder WebVR (A-Frame), Unity en Unreal Engine. Maar ook naar de verschillende soorten ervaringen zoals: Augmented Reality, 360\textdegree\space en VR met en zonder bewegingsvrijheid. 
Om te bepalen welke technologie de beste is zal er eerst een longlist worden opgesteld met de mogelijke technologieën. Vanuit deze longlist zal er dan een shortlist worden opgesteld met de meest belovende technologieën. Dit zijn degene die realistisch zouden zijn om te implementeren (huidige vooruitgang technologie en beperking budget). Aan de hand van elke technologie in de shortlist wordt er dan gekeken hoe goed de ervaring is voor de gebruiker. Dit zal worden gemeten aan de hand van functionele en niet-functionele requirements waaronder: de instapkost, de interactiviteit voor de gebruiker en bruikbaarheid binnenin de sector.

%---------- Verwachte resultaten ----------------------------------------------
\section{Verwachte resultaten}
\label{sec:verwachte_resultaten}

Als resultaat wordt er een analyse verwacht van de verschillende technologieën die hun waarde (instapkost, interactiviteit en bruikbaarheid in de realiteit) met elkaar vergelijkt. Uit deze analyse zal het dan mogelijk zijn om af te leiden welke ervaring de beste is voor de gekozen use case. De beste technologie zal dan verder ontwikkeld worden om zo een beter beeld te hebben over de volledige ervaring.

%---------- Verwachte conclusies ----------------------------------------------
\section{Verwachte conclusies}
\label{sec:verwachte_conclusies}

Er wordt verwacht dat uit de analyse van de verschillende technologieën gaat blijken dat hoe hoger de instapkost is hoe hoger de interactiviteit zal zijn. Dit betekent niet dat het prototype met de hoogste instapkost en dus hoogste interactiviteit ook de hoogste bruikbaarheid zal hebben. Er wordt verwacht dat augmented reality de hoogste bruikbaarheid zal hebben door de lage instapkost maar dat de interactiviteit wel beperkt zal zijn. Een echte volledige virtual reality applicatie zal dan veel interactiever zijn maar ook wel de hoogste instapkost hebben. Alsook wordt er verwacht dat de ontwikkeling van een applicatie met WebVR het makkelijkste gaat zijn door zijn simpliciteit en gelijkenis met HTML. De beste technologie zal dus diegene zijn die een goede balans kan vinden tussen instapkost en interactiviteit.